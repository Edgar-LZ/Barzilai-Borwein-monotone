\documentclass[10pt,a4paper]{article}

\usepackage{geometry}
\usepackage{graphicx}
\usepackage[utf8]{inputenc}
\usepackage{enumitem}
\usepackage{ dsfont }
\usepackage{lipsum}
\usepackage{ amssymb }
\usepackage{amsmath}
%\usepackage[spanish,es-tabla]{babel}
\usepackage{listings}
\usepackage{xcolor}
\usepackage{algorithmicx}
\usepackage{algorithm}
\usepackage{algpseudocode}
\usepackage{datetime}
\usepackage{booktabs}
\usepackage{amsmath}
\usepackage{subfigure}
\newcommand*{\defeq}{\stackrel{\text{def}}{=}}

\geometry{
	a4paper,
	left=20mm,
	top=10mm,
}


\newenvironment{QandA}{\begin{enumerate}[label=\bfseries\arabic*.]}
	{\end{enumerate}}
\newenvironment{answered}{\setlength{\parindent}{2em}\par\rule{0.5in}{1pt}\par\normalfont}{}


\definecolor{codegreen}{rgb}{0,0.6,0}
\definecolor{codegray}{rgb}{0.5,0.5,0.5}
\definecolor{codepurple}{rgb}{0.58,0,0.82}
\definecolor{backcolour}{rgb}{0.95,0.95,0.92}


\lstdefinestyle{mystyle}{
	backgroundcolor=\color{backcolour},   
	commentstyle=\color{codegreen},
	keywordstyle=\color{magenta},
	numberstyle=\tiny\color{codegray},
	stringstyle=\color{codepurple},
	basicstyle=\ttfamily\footnotesize,
	breakatwhitespace=false,         
	breaklines=true,                 
	captionpos=b,                    
	keepspaces=true,                 
	numbers=left,                    
	numbersep=5pt,                  
	showspaces=false,                
	showstringspaces=false,
	showtabs=false,                  
	tabsize=2
}



\title{\textbf{Tema de proyecto}}
\date{\today}

\author{Giovanni Gamaliel López Padilla \\ Edgar Osvaldo López Zúñiga}

\begin{document}
\maketitle
\vspace{-0.4in}
\noindent\rule{\textwidth}{0.5pt}

\par El tema elegido para el proyecto es Aceleración del método de Barzilai-Borwein. 
\par Utilizaremos como base el artículo "On the acceration of the Barzilai-Borwein method" de Huang, et al. En donde se propone un nuevo tamaño de paso para acelerar el método al minimizar funciones cuadráticas bidimensionales fuertemente convexas. Utilizando este tamaño de paso se desarrollan técnicas de búsqueda lineal que toman de forma adaptable los tamaños de paso no-monótonos del método de Barzilai-Borwein y tamaños de paso monótonos para minimizar las funciones.
\par Nuestro objetivo es implementar el método de Barzilai-Borwein original, así como la versión presentada en el artículo incluyendo el tamaño de paso monótono y comparar resultados numéricos obtenidos así como el número de pasos hasta la convergencia para distintos puntos iniciales y funciones que cumplan con las características previamente mencionadas.
\end{document}
