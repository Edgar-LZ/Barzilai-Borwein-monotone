\section*{Resultados}

Se implementaron los métodos de Barzilai estabilizado (BB), descenso de gradiente con bisección (SD), ANGM, ANGR1 y ANGR2. Los métodos de busqueda antes mencionados fueron probados con las siguientes funciones:

\subsection*{Rosembrock}

La función de Rosembrock se define en la ecuación \ref{eq:rosembrock}.

\begin{equation}
    f(x) = \sum_{i=1}^{n-1}  100(x_{i+1}-x_{i}^2)^2 +(1-x_i)
    \label{eq:rosembrock}
\end{equation}

Su gradiente es calculado de la siguiente manera:

\begin{equation}
    \nabla f (x) =\begin{cases}
        -400x_i(x_{i+1}-x_{i}^2)                              & \text{para } i=1   \\[0.25cm]
        200(x_{i}-x_{i-1}^2-400x_i(x_{i+1}-x_{i}^2) -2(1-x_i) & \text{para } 1<i<n \\[0.25cm]
        200(x_{i}-x_{i-1}                                     & \text{para } i=n
    \end{cases} \label{eq:rosembrock_gradient}
\end{equation}


y el hessiano se puede obtener realizando la siguiente operación:

\begin{equation}
    \nabla^2 f(x)  = \begin{cases}
        \ddpartial{f}{x_i} = -200                    & \text{para } x=1         \\[0.25cm]
        \ddpartial{f}{x_i} =1200x_i^2-400x_{i+1}+202 & \text{para } 1<i<n       \\[0.25cm]
        \dpartiald{f}{x_i}{x_{i+1}}  = -400x_i       & \text{para } 0\leq i < n \\[0.25cm]
        \ddpartial{f}{x_i} = 200                     & \text{para } i=n
    \end{cases}
    \label{eq:rosembrock_hessian}
\end{equation}

\subsection*{Wood}

La función de Wood se define en la ecuación \ref{eq:wood}.

\changefontsizes{10pt}
\begin{equation}
    f(x) = 100(x_1^2-x_2)^2+(x_1-1)^2+(x_3-1)^2+90(x_3-x_4)^2 +10.1((x_2-1)^2+(x_4-1)^2)+19.8(x_2-1)(x_4-1) \label{eq:wood}
\end{equation}
\normalsize

donde $x\in \Real^4$. Con la función de Wood definida, podemos obtener le gradiente de la función de Wood. El resultado del gradiente de la función de Wood se encuentra en la siguiente ecuación.

\begin{equation}
    \nabla f(x) = \begin{cases}
        \dpartial{f}{x_1}=400(x_1^2-x_2)x_1 +2(x_1-1)             \\[0.25cm]
        \dpartial{f}{x_2}=-200(x_1^2-x_2)+20.2(x_2-1)+19.8(x_4-1) \\[0.25cm]
        \dpartial{f}{x_3}=2(x_3-1)+360(x_3^2-x_4)x_3              \\[0.25cm]
        \dpartial{f}{x_4}=-180(x_3^2-x_4)+20.2(x_4-1)+19.8(x_2-1)
    \end{cases} \label{eq:wood_gradient}
\end{equation}

De igual forma, se puede obtener el hessiano de la función de Wood. El resultado del Hessiano se calculado en la siguiente ecuación.

\begin{equation}
    \nabla^2 f(x) = \begin{cases}
        \ddpartial{f}{x_1} = 400(x_1^2-x_2)+800x_1^2+2             \\[0.25cm]
        \dpartiald{f}{x_1}{x_2} =\dpartiald{f}{x_2}{x_1} = -400x_1 \\[0.25cm]
        \ddpartial{f}{x_2} = 220.2                                 \\[0.25cm]
        \dpartiald{f}{x_4}{x_2} = \dpartiald{f}{x_2}{x_4} = 19.8   \\[0.25cm]
        \ddpartial{f}{x_3} = 720x_3^2+360(x_3^2-x_4)+2             \\[0.25cm]
        \dpartiald{f}{x_4}{x_3}=\dpartiald{f}{x_3}{x_4} = -360x_3  \\[0.25cm]
        \ddpartial{f}{x_4} = 200.2
    \end{cases} \label{eq:wood_hessian}
\end{equation}

\subsection*{Lambda}

En el artículo de Yakui Huang\cite{huang_2022_1} se utiliza una función definida de la siguiente manera:

\begin{equation*}
    f(x) = \frac{1}{2}x^TAx \qquad A=\begin{cases}
        0                           & \text{para } i\neq 0 \\
        10^{\frac{ncond}{n-1}(n-j)} & \text{para } i=j
    \end{cases}
\end{equation*}

donde $ncond=log \kappa$ con $\kappa = 10^3$ y $n=10$. Calculando su gradiente se obtiene que tiene la siguiente forma

\begin{equation*}
    \nabla f(x) = Ax
\end{equation*}

por consecuente, el hessiano de la función es unicamente la matriz $A$.

\subsection*{Cuadratica de la forma diag\{1,$\lambda$\}}

En el mismo artículo\cite{huang_2022_1} se propone una función cuadratica donde la matriz esta definida de la siguiente manera:

\begin{equation*}
    A = diag\{1,\lambda\}
\end{equation*}

En nuestro caso tomamos $\lambda=10$. El gradiente y hessiano de esta función tienen la misma forma que la función lambda.

\subsection*{Porcentaje de contribución}

Se definió una función $\gamma$ para medir el porcentaje de las dos componentes más grandes del gradiente en cada iteración de la optimización. La función $\gamma$ tiene la siguiente forma

\begin{equation*}
    \gamma(g_k) = \frac{|g_k^{(1)}|+|g_k^{(2)}|}{\sum\limits_i |g_k^{(i)}|}
\end{equation*}

Realizando el calculo de la función $\gamma$ con la función lambda en el punto inicial $x=(10,10,\dots,10)^T$ se obtuvieron los resultados mostrados en la figura \ref{fig:gamma} para los diferentes métodos.

\begin{figure}[H]
    \begin{subfigure}{8.4cm}
        \includegraphics[width=1\textwidth]{graphics/gamma/steepest.png}
        \caption{Método de descenso de gradiente.}
    \end{subfigure}
    \begin{subfigure}{8.4cm}
        \includegraphics[width=1\textwidth]{graphics/gamma/barzilai.png}
        \caption{Método BB.}
    \end{subfigure}
    \begin{subfigure}{8.4cm}
        \includegraphics[width=1\textwidth]{graphics/gamma/ANGM.png}
        \caption{Método ANGM.}
    \end{subfigure}
    \begin{subfigure}{8.4cm}
        \includegraphics[width=1\textwidth]{graphics/gamma/ANGR1.png}
        \caption{Método ANGR1}
    \end{subfigure}
    \begin{subfigure}{16.5cm}
        \centering
        \includegraphics[width=8.4cm]{graphics/gamma/ANGR1.png}
        \caption{Método ANGR2.}
    \end{subfigure}
    \caption{Función $\gamma$ para diferentes métodos (izquierda) y la posición en el vector de la componente más grande (derecha).}
    \label{fig:gamma}
\end{figure}

En la tabla \ref{table:gamma_function} se muestra el número de iteraciones en las que se obtuvo un valor mayor a 0.8 para la función $\gamma$ en cada métodode optimización.

\subsection*{Porcentaje de contribución}

Se definió una función $\gamma$ para medir el porcentaje de las dos componentes más grandes del gradiente en cada iteración de la optimización. La función $\gamma$ tiene la siguiente forma

\begin{equation*}
    \gamma(g_k) = \frac{|g_k^{(1)}|+|g_k^{(2)}|}{\sum\limits_i |g_k^{(i)}|}
\end{equation*}

Realizando el calculo de la función $\gamma$ con la función lambda en el punto inicial $x=(10,10,\dots,10)^T$ se obtuvieron los resultados mostrados en la figura \ref{fig:gamma} para los diferentes métodos.

\begin{figure}[H]
    \begin{subfigure}{8.4cm}
        \includegraphics[width=1\textwidth]{graphics/gamma/steepest.png}
        \caption{Método de descenso de gradiente.}
    \end{subfigure}
    \begin{subfigure}{8.4cm}
        \includegraphics[width=1\textwidth]{graphics/gamma/barzilai.png}
        \caption{Método BB.}
    \end{subfigure}
    \begin{subfigure}{8.4cm}
        \includegraphics[width=1\textwidth]{graphics/gamma/ANGM.png}
        \caption{Método ANGM.}
    \end{subfigure}
    \begin{subfigure}{8.4cm}
        \includegraphics[width=1\textwidth]{graphics/gamma/ANGR1.png}
        \caption{Método ANGR1}
    \end{subfigure}
    \begin{subfigure}{16.5cm}
        \centering
        \includegraphics[width=8.4cm]{graphics/gamma/ANGR1.png}
        \caption{Método ANGR2.}
    \end{subfigure}
    \caption{Función $\gamma$ para diferentes métodos (izquierda) y la posición en el vector de la componente más grande (derecha).}
    \label{fig:gamma}
\end{figure}

En la tabla \ref{table:gamma_function} se muestra el número de iteraciones en las que se obtuvo un valor mayor a 0.8 para la función $\gamma$ en cada métodode optimización.

\subsection*{Porcentaje de contribución}

Se definió una función $\gamma$ para medir el porcentaje de las dos componentes más grandes del gradiente en cada iteración de la optimización. La función $\gamma$ tiene la siguiente forma

\begin{equation*}
    \gamma(g_k) = \frac{|g_k^{(1)}|+|g_k^{(2)}|}{\sum\limits_i |g_k^{(i)}|}
\end{equation*}

Realizando el calculo de la función $\gamma$ con la función lambda en el punto inicial $x=(10,10,\dots,10)^T$ se obtuvieron los resultados mostrados en la figura \ref{fig:gamma} para los diferentes métodos.

\begin{figure}[H]
    \begin{subfigure}{8.4cm}
        \includegraphics[width=1\textwidth]{graphics/gamma/steepest.png}
        \caption{Método de descenso de gradiente.}
    \end{subfigure}
    \begin{subfigure}{8.4cm}
        \includegraphics[width=1\textwidth]{graphics/gamma/barzilai.png}
        \caption{Método BB.}
    \end{subfigure}
    \begin{subfigure}{8.4cm}
        \includegraphics[width=1\textwidth]{graphics/gamma/ANGM.png}
        \caption{Método ANGM.}
    \end{subfigure}
    \begin{subfigure}{8.4cm}
        \includegraphics[width=1\textwidth]{graphics/gamma/ANGR1.png}
        \caption{Método ANGR1}
    \end{subfigure}
    \begin{subfigure}{16.5cm}
        \centering
        \includegraphics[width=8.4cm]{graphics/gamma/ANGR1.png}
        \caption{Método ANGR2.}
    \end{subfigure}
    \caption{Función $\gamma$ para diferentes métodos (izquierda) y la posición en el vector de la componente más grande (derecha).}
    \label{fig:gamma}
\end{figure}

En la tabla \ref{table:gamma_function} se muestra el número de iteraciones en las que se obtuvo un valor mayor a 0.8 para la función $\gamma$ en cada métodode optimización.

\input{sections/results/tables/gamma.tex}

\begin{table}[H]
    \small
    \centering
    \begin{tabular}{llrrrrr}
        \hline
        \textbf{Función} & \textbf{Valor} & \textbf{SD} & \textbf{BB} & \textbf{ANGM} & \textbf{ANGR1} & \textbf{ANGR2} \\[0.1cm]\hline
                         & Función        & 0.00        & 0.00        & 0.00          & 0.00           & 0.00           \\
        Lambda           & Gradiente      & 0.00        & 0.00        & 0.00          & 0.00           & 0.00           \\
                         & Iteraciones    & 7764.81     & 2770.34     & 252.43        & 289.49         & 255.83         \\\hline
                         & Función        & 0.00        & 0.00        & 0.00          & 0.00           & 0.00           \\
        Cuadrática       & Gradiente      & 0.00        & 0.00        & 0.00          & 0.00           & 0.00           \\
                         & Iteraciones    & 713.00      & 1531.23     & 5.24          & 7.36           & 9.70           \\\hline
                         & Función        & 4.16        & 42.25       & 0.00          & 0.00           & 0.00           \\
        Rosembrock       & Gradiente      & 0.57        & 45.45       & 0.00          & 0.00           & 0.00           \\
                         & Iteraciones    & 10000.00    & 1603.21     & 68.84         & 70.17          & 67.15          \\\hline
                         & Función        & 0.00        & 3.16        & 5.07          & 4.56           & 8.43           \\
        Wood             & Gradiente      & 0.00        & 0.00        & 0.00          & 0.00           & 0.00           \\
                         & Iteraciones    & 9428.69     & 1806.43     & 360.36        & 292.43         & 199.21         \\\hline
    \end{tabular}
    \caption{Media de las 100 ejecucciones partiendo de puntos aleatorios.}
    \label{table:mean}
\end{table}

\begin{table}[H]
    \changefontsizes{11pt}
    \centering
    \begin{tabular}{llrrrrr}
        \hline                                                                                                          \\
        \textbf{Función} & \textbf{Valor} & \textbf{SD} & \textbf{BB} & \textbf{ANGM} & \textbf{ANGR1} & \textbf{ANGR2} \\[0.1cm]\hline
        \\
                         & Función        & 0.000000    & 0.000000    & 0.000000      & 0.000000       & 0.000000       \\[0.25cm]
        Lambda           & Gradiente      & 0.000000    & 0.000000    & 0.000000      & 0.000000       & 0.000000       \\[0.25cm]
                         & Iteraciones    & 612.385412  & 209.340437  & 32.553066     & 28.338322      & 25.043016      \\[0.25cm]\hline
        \\
                         & Función        & 0.000000    & 0.000000    & 0.000000      & 0.000000       & 0.000000       \\[0.25cm]
        Cuadrática       & Gradiente      & 0.000000    & 0.000000    & 0.000000      & 0.000000       & 0.000000       \\[0.25cm]
                         & Iteraciones    & 43.810658   & 96.111137   & 1.198652      & 1.856030       & 2.886751       \\[0.25cm]\hline
        \\
                         & Función        & 14.244453   & 393.585673  & 0.000000      & 0.000000       & 0.000000       \\[0.25cm]
        Rosembrock       & Gradiente      & 1.357743    & 396.058464  & 0.000000      & 0.000000       & 0.000000       \\[0.25cm]
                         & Iteraciones    & 0.000000    & 1867.607211 & 17.317429     & 20.596192      & 14.927011      \\[0.25cm]\hline
        \\
                         & Función        & 0.000000    & 10.092718   & 12.223427     & 11.860350      & 14.404998      \\[0.25cm]
        Wood             & Gradiente      & 0.000019    & 0.000000    & 0.000000      & 0.000000       & 0.000000       \\[0.25cm]
                         & Iteraciones    & 465.597882  & 633.528762  & 190.558643    & 204.762156     & 119.763512     \\[0.25cm]\hline
    \end{tabular}
    \caption{Desviaciones estandar de las 100 ejecucciones partiendo de puntos aleatorios.}
    \label{table:deviation}
\end{table}