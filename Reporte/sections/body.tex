\section*{Desarrollo}
En 1988 Barzilai y Borwein propusieron dos tamaños de paso para mejorar el desempeño de métodos de descenso de gradiente, estos tamaños de paso son los siguientes:
$$
\alpha^{BB1}_k = \frac{s^T_{k-1}s_{k-1}}{s^T_{k-1}y_{k-1}} \text{ y } \alpha_k^{BB2} = \frac{s_{k-1}^Ty_{k-1}}{y_{k-1}^Ty_{k-1}}
$$
donde $s_{k-1} = x_k - x_{k-1}$ y $y_{k-1} = g_k - g_{k-1}$. Ahora, si consideramos por la desigualdad de Cauchy-Schwarz que $\|s_{k-1}\|^2\|y_{k-1}\|^2 \geq (s_{k-1}^Ty_{k-1})^2$ se puede observar que cuando $s_{k-1}^Ty_{k-1}>0$ se cumple que $\alpha_k^{BB1} \geq \alpha_k^{BB2}$. Por lo que se suele llamar al paso $\alpha_k^{BB1}$ el paso largo de Barzilai-Borwein y a $\alpha_k^{BB2}$ se le llama paso corto de Barzilai-Borwein.
