\section*{Desarrollo}
En 1988 Barzilai y Borwein propusieron dos tamaños de paso para mejorar el desempeño de métodos de descenso de gradiente, estos tamaños de paso son los siguientes:
$$
\alpha^{BB1}_k = \frac{s^T_{k-1}s_{k-1}}{s^T_{k-1}y_{k-1}} \text{ y } \alpha_k^{BB2} = \frac{s_{k-1}^Ty_{k-1}}{y_{k-1}^Ty_{k-1}}
$$
donde $s_{k-1} = x_k - x_{k-1}$ y $y_{k-1} = g_k - g_{k-1}$. Ahora, si consideramos por la desigualdad de Cauchy-Schwarz que $\|s_{k-1}\|^2\|y_{k-1}\|^2 \geq (s_{k-1}^Ty_{k-1})^2$ se puede observar que cuando $s_{k-1}^Ty_{k-1}>0$ se cumple que $\alpha_k^{BB1} \geq \alpha_k^{BB2}$. Por lo que se suele llamar al paso $\alpha_k^{BB1}$ el paso largo de Barzilai-Borwein y a $\alpha_k^{BB2}$ se le llama paso corto de Barzilai-Borwein. En el caso en el que la función es cuadrática, el paso $\alpha_k^{BB1}$ será el tamaño de paso de máximo descenso con un retardo de un paso y $\alpha_k^{BB2}$ será el paso del método de mínimo gradiente.
\par Se ha probado que el método de Barzilai-Borwein (BB) tiene convergencia R-superlineal para minimizar funciones cuadráticas bidimensionales fuertemente convexas y R-lineal para el caso general n-dimensional.
\par El método de BB cuenta con la propiedad de reducir los valores de la función objetivo de manera no monóntona, y esta propiedad es una característica intrínseca del mismo que causa su eficiencia. Sin embargo, es importante para los métodos de gradiente mantener la monotonicidad.
\par Debido a que en el caso de optimización general es difícil calcular un tamaño de paso $\alpha_k^{SD}$ y a que el algoritmo de BB ha sido exitoso a la hora de mejorar el desmpeño de los métodos de gradiente, el trabajo analizado se motiva en la búsqueda de una forma de aceleración para el método de Barzilai-Borwein incorporando pasos monótonos. 
\par Considérese primero el problema de acelerar los métodos de descenso de gradiente que generan secuencias de iterados de la siguiente forma:
$$
	x_{k+1} = x_k - \alpha_kg_k
$$
para resolver el problema 
$$
	\min_{x\in\mathbb{R}^n}f(x)
$$
donde $f:\mathbb{R}^n \rightarrow \mathbb{R}^n$ es continuamente diferenciable, $g_k=\nabla f(x_k)$ y $\alpha_k>0$ es el tamaño de paso sobre la dirección del gradiente.
Queremos optimizar una función cuadrática de la forma:
$$
	f(x) = \frac{1}{2}x^TAx- b^Tx
$$
donde $b\in\mathbb{R}^n$ y $A\in \mathbb{R}^{n\times n}$ utilizando el siguiente tamaño de paso:
$$
	\alpha_k(\Psi(A)) = \frac{g_{k-1}^Tg_{k-1}}{g^T_{k-1}\Psi(A)Ag_{k-1}}
$$
En donde $\Psi(\cdot)$ es una función real analítica en $[\lambda_1, \lambda_n]$ que se puede expresar como:
$$
	\Psi(z) = \sum\limits_{k=-\infty}^{\infty} c_kz^k,\ c_k\in \mathbb{R}^n
$$
tal que $0<\sum_{k =- \infty}^{\infty}c_k z^k<+\infty$ para todos los $z\in[\lambda_1, \lambda_n]$, donde $\lambda_1$ y $\lambda_n$ son los eigenvalores más pequeño y más grande respectivamente. Se puede observar que los dos tamaños de paso de Barzilai-Borwein $\alpha_k^{BB1}$ y $\alpha_k^{BB2}$ se pueden obtener de este tamaño de paso más general haciendo $\Psi(A) = I$ y $\Psi(A) = A$ en la expresión para $\alpha_k$ respectivamente.
\par Más adelante se procederá a describir el tamaño de paso $\tilde{\alpha}_k$ que se deriva y se utiliza en el trabajo para minimizar funciones cuadráticas bidimensionales convexas en un máximo de 5 iteraciones. También se hablará de modificaciones y consideraciones que se hacen sobre este tamaño de paso para proponer métodos adaptativos no monótonos de gradiente que utilizan los pasos largo y corto del método de BB junto con algunos pasos montonos usando $\tilde{\alpha}_k(A)$.
\par A su vez, se discutirán dos variaciones del método desarrollado para minimizar de forma eficiente funciones más genertales utilizando tamaños de paso con retardo.

\subsection*{Tamaño de paso}
En esta sección se discutirá sobre el tamaño de paso derivado en el trabajo para minimizar funciones cuadráticas bidimiensionales fuertemente convexas bajo la motivación de requerir una terminación finita.
\par Es necesario observar que la propiedad de que dos gradientes consecutivos generados por el método de máximo descenso, no se mantiene para métodos que utilizan el tamaño de paso general antes definido. También se observa que este método es invariante ante tralaciones y rotaciones cuando se minimizan funciones cuadráticas, por lo que se puede asumir para simplicidad que la matriz $A$ en la definición que dimos para las funciones cuadráticas es una matriz de la forma 
$$
	A = \{\lambda_1, ..., \lambda_n\},
$$
donde $0<\lambda_1\leq...\leq\lambda_n$.

\par Se ha mostrado que una familia de métodos de gradiente entre los que se incluye máximo descenso y mínimo gradiente reducirán asintóticamente sus búsqiedas en un subespacio bidimensional y que se podrían explotar algunas propiedades ortogonales en este subespacio bidimensional para acelerar los métodos. Más adelante se discutirá un caso en el que ocurre que para una función cuadrática, las búsquedas del método de Barzilai-Borwein con tamaño de paso $\alpha_k^{BB1}$ son dominadas en subespacios bidimensionales. Haciendo uso de este conocimiento, se quiere acelerar la convergencia de los métodos con tamaño de paso $\alpha_k$ como el paso general que se propuso anteriormente en un espacio de menos dimensiones si se mantienen propiedades ortogonales.
\par Suponiendo que para un $k>0$, existe un $q_k$ que satisface
$$
	(I-\alpha_{k-1}A)q_k = g_{k-1}
$$
Esta $q_k$ es también invariante ante rotaciones y traslaciones, por lo que se puede seguir asumiendo que la matriz $A$ de la función cuadrática es diagonal. Con esto, se tiene el siguiente lema que presenta una propiedad importante para la derivación del tamaño de paso.\par
\bf Lema 1 (Propiedad Ortogonal) \normalfont Suponiendo que la secuencia $\{g_k\}$ se obtiene aplicando un método de gradiente con pasos como el paso general $\alpha_k)$ para minimizar una función cuadrática y $q_k$ satisface que $(I-\alpha_{k-1}A)q_k = g_{k-1}$. Entonces
$$
	q_k^T\Psi(A)g_{k+1} = 0.
$$
\it Prueba \normalfont Utilizando la forma en la que se definió $q_k$ y la expresión para el tamaño de paso general $\alpha_k$, se tiene que:
\begin{align*}
	q_k^T \Psi(A)g_{k+1} &= q_k^T\Psi(A)(I-\alpha_kA)(I-\alpha_{k-1}A)g_{k-1}\\
	&= q_k^T(I-\alpha_kA)\Psi(A)(I-\alpha_{k-1}A)g_{k-1}\\
	&= g_{k-1}^T\Psi(A)(I-\alpha_kA)g_{k-1}\\
	&=g_{k-1}^T\Psi(A)g_{k-1}-\alpha_kg_{k-1}^T\Psi(A)Ag_{k-1}\\
	&=g_{k-1}^T\Psi(A)g_{k-1}\left[\frac{g_{k-1}^T\Psi(A)g_{k-1}}{g_{k-1}^T\Psi(A)Ag_{k-1}}-\alpha_k\right] = 0
\end{align*}
con lo que se demuestra que el vector $q_k^T$ y $g_{k+1}$ son ortogonales bajo $\Psi(A)$.
\par Ahora se busca derivar un tamaño de paso que al combinarlo con el método de gradiente general se puede alcanzar una terminación finita para minimizar funciones bidimensionales fuertemente convexas.
\par Por el Lema 1 se tiene que $g_k^T$  y $q_{k-1}$ son ortogonales bajo $\Psi(A)$ para cualquier $k>0$. Ahora suponga que los vectores $\Psi^r(A)q_{k-1}$ y $\Psi^{1-r}(A)g_k$ son vectores no nulo, con $r\in \mathbb(R)$. Y consideremos el problema de minimizar una función $f$ en un subespacio bidimensional generado por $ u = \frac{\Psi^r(A)q_{k-1}}{\|\Psi^r(A)q_{k-1}\|}$ y $v = \frac{\Psi^r(A)g_{k}}{\|\Psi^r(A)g_{k}\|}$ y sea
$$
\phi(t,l):=f\left(x_k+t\frac{\Psi^r(A)q_{k-1}}{\|\Psi^r(A)q_{k-1}\|} + l \frac{\Psi^r(A)g_{k}}{\|\Psi^r(A)g_{k}\|}\right)
$$
que al expandir utilizando una serie de Taylor nos lleva a
\begin{align*}
	\phi(t,l) = f(x_k)+\nabla^T f(x_k) \left[tu+lv \right] + \frac{1}{2}\left[tu+ lv \right]^T\nabla^2 f(x_k) \left[tu+lv \right] 
\end{align*}
Ahora, si definimos la matriz $B_k$ como
$$
	B_k =  \left(u, v\right)^T
$$
y recordando que $u$ y $v$ forman una base ortogonal para $\mathbb{R}^2$ podemos reescribir la expansión en serie de Taylor de $\phi(t, l)$ como sigue:
$$
\phi(t, l) =  f(x_k)+g_k^TB_k^T\begin{pmatrix}t\\l\end{pmatrix} +\frac{1}{2}\begin{pmatrix}t\\l\end{pmatrix}^TB_k A B_k^T\begin{pmatrix}t\\l\end{pmatrix}
$$
Ahora, podemos definir
$$
\vartheta_k = B_kg_k = \begin{pmatrix}\frac{g_k^T\Psi^r(A)q_{k-1}}{\|\Psi^{r}(A)q_{k-1}\|}\\
\frac{g_k^T\Psi^{1-r}(A)g_k}{\|\Psi^{1-r}(A)g_k\|}\end{pmatrix}
$$
y 
$$
H_k = B_kAB_k^T =\begin{pmatrix}
	\frac{q_{k-1}^T\Psi^{2r}(A)Aq_{k-1}}{\|\Psi^{r}(A)q_{k-1}\|^2} & \frac{q_{k-1}^T\Psi^r(A)Ag_{k}}{\|\Psi^{r}(A)q_{k-1}\|\Psi^{1-r}(A)g_k\| }\\
	\frac{q_{k-1}^T\Psi^r(A)Ag_{k}}{\|\Psi^{r}(A)q_{k-1}\|\Psi^{1-r}(A)g_k\| } & 
	\frac{g_k^T\Psi^{2(1-r)}(A)Ag{k}}{\|\Psi^{r}(A)g_k\|^2} 
\end{pmatrix} 
$$
con lo que la expresión para $\phi(t, l)$ se convertiría en
$$
\phi(t, l) =  f(x_k)+\vartheta_k^T\begin{pmatrix}t\\l\end{pmatrix} +\frac{1}{2}\begin{pmatrix}t\\l\end{pmatrix}^T H_k\begin{pmatrix}t\\l\end{pmatrix}
$$
Denotando las componentes de $H_k$ por $H_k^(ij)$, $i,j = 1,2$ y notando que $B_kB^T_k = I$ se tiene el siguiente teorema de terminación finita
\par \bf Teorema 1 (Terminación finita) \normalfont Suponga que un método de gradiente se aplica para minimizar una función cuadrática bidimensional con $\alpha_k$ dado por el paso general mencionado anteriormente para todas las $k\neq k_0$ y usa el tamaño de paso
$$
\tilde{\alpha}_{k_0} = \frac{2}{\left(H^{(11)}_{k_0} + H^{(22)}_{k_0}\right)+\sqrt{\left(H^{(11)}_{k_0} - H^{(22)}_{k_0}\right)^2 } + 4\left(H_{k_0}^{(12)}\right)^2}
$$
en la iteración $k_0$-ésima donde $k_0\geq 2$. Entonces, el método encontrará el minimizador en máximo $k_0+3$ iteraciones.

\it Prueba \normalfont Supongamos que $x_k$ no es un minimizador para toda $k=1, ..., k_0+1$. Para simplificar durante la prueba, se utilizará comom notación $k$ para referirse a $k_0$.
\par Es necesario observar que $\tilde{\alpha_k}$ satisface la siguiente ecuación cuadrática
$$
	\tilde{\alpha}^2_k\Delta-\tilde{\alpha}_k\left(H^(11)_k + H_k^(22)\right) + 1 = 0
$$
en donde $\Delta = \det{(H_k)}$ = $\det{(A)}\>0$. Ahora, sea
$$
\Theta = \left(H_k^{(12)}\vartheta_k^{(1)}+ H_k^{(22)}\vartheta^{(2)}_k\right)\vartheta_k^{(1)} -  \left(H_k^{(11)}\vartheta_k^{(1)}+ H_k^{(12)}\vartheta^{(2)}_k\right)\vartheta_k^{(2)} 
$$
en donde $\vartheta_k^{(i)}$ son las componentes de $\vartheta_k$. Multiplicando $\Theta$ a la ecuación cuadrática para $\alpha$ se tiene lo siguiente
$$
\tilde{\alpha}_k^2\Delta\Theta-\tilde{\alpha_k}\left(H_k^{(11)}+H_k^{(12)}\right)\Theta + \Theta = 0,
$$
que exactamente es
\begin{align*}
	(H_k^{(22)}v_k^{1}-H_k^{(12)}v_k^{(2)}-\tilde{\alpha}_k\Delta\vartheta_k^{(1)})[\vartheta_k ^{(2)}-\tilde{\alpha}_k(H_k^{(12)}\vartheta_k^{(1)}+H_k^{(22)}\vartheta_k^{(2)})]\\
	= (H_k^{(11)}v_k^{2}-H_k^{(12)}v_k^{(1)}-\tilde{\alpha}_k\Delta\vartheta_k^{(1)})[\vartheta_k ^{(1)}-\tilde{\alpha}_k(H_k^{(11)}\vartheta_k^{(1)}+H_k^{(12)}\vartheta_k^{(2)})].
\end{align*}
Como paréntesis consideremos dos vectores $\boldsymbol{A} = (a_1, a_2)^T$ y $\boldsymbol{B} = (b_1, b_2)^T$, y supongamos que se cumple que $a_1b_2 = a_2b_1$, sustituyendo $b_2$ en $\boldsymbol{B}$, tenemos lo siguiente
$$
	\boldsymbol{B} = \begin{pmatrix}b_1\\\frac{b_1a_2}{a_1}\end{pmatrix} = \frac{b_1}{a_1}\boldsymbol{A}
$$
por los que podemos decir que $\boldsymbol{B}$ es paralelo a $\boldsymbol{A}$.
\par Haciendo uso del resultado anterior podemos decir que el vector 
$$
\begin{pmatrix}
	H_k^{(22)}\vartheta_k^{(1)}-H_k^{(12)}\vartheta_k^{(2)}-\tilde{\alpha}_k\Delta\vartheta_k^{(1)}\\
	H_k^{(11)}\vartheta_k^{(2)}-H_k^{(12)}\vartheta_k^{(1)}-\tilde{\alpha}_k\Delta\vartheta_k^{(2)}
\end{pmatrix}
$$
es paralelo a 
$$
	\begin{pmatrix}
		v_k^{(1)} - \tilde{\alpha}_k(H_k^{(11)}\vartheta_k^{(1)}+H_k^{(12)}\vartheta_k^{(2)})\\
		v_k^{(2)} - \tilde{\alpha}_k(H_k^{(12)}\vartheta_k^{(1)}+H_k^{(22)}\vartheta_k^{(2)})
	\end{pmatrix}
$$
que escrito de otra forma significa que 
$$
\vartheta_k+H_k(-\tilde{\alpha}_k\vartheta_k) \text{ es paralelo a } H_k^{-1}\vartheta_j-\tilde{\alpha}_k\vartheta_k
$$
es decir
$$
\vartheta_k+H_k(-\tilde{\alpha}_k\vartheta_k) = \gamma(H_k^{-1}\vartheta_j-\tilde{\alpha}_k\vartheta_k)
$$
en donde $\gamma$ es una constante diferente de cero. Ahora, si multiplicamos a ambos lados de la ecuación por $B_k^T$ por la derecha se tiene
$$
B_k^T[\vartheta_k+H_k(-\tilde{\alpha}_k\vartheta_k)] = \gamma B_k^T(H_k^{-1}\vartheta_j-\tilde{\alpha}_k\vartheta_k).
$$
Se sabe que $B^T_kB_k = I$, $\vartheta_k = B_kg_k$ y que $H_k = B_kAB_k^T$. Además $g_{k+1} = g_k+\tilde{\alpha}_kAg_k$. Haciendo uso de este conocimiento, se llega a que 
$$
g_k+A(-\tilde{\alpha}_kB^T_k\vartheta_k) = \gamma(B_k^TH_k^{-1}\vartheta_k+\tilde{\alpha}_kg_k)
$$
$$
g_k+A(-\tilde{\alpha}_kg_k) = \gamma(B_k^TH_k^{-1}\vartheta_k+\tilde{\alpha}_kg_k)
$$
$$
g_{k+1} = \gamma(B_k^TH_k^{-1}\vartheta_k+\tilde{\alpha}_kg_k)
$$
$$
g_{k+1} = \gamma(B_k^TB_kA^{-1}B_k^T\vartheta_k+\tilde{\alpha}_kg_k)
$$
$$
g_{k+1} = \gamma(A^{-1}g_k+\tilde{\alpha}_kg_k)
$$
Ahora, factorizando $A^{-1}$ se tiene
$$
g_{k+1} = \gamma A^{-1}(g_k+\tilde{\alpha}_kAg_k)
$$
de nuevo, considerando que $g_{k+1} = g_k+\tilde{\alpha}_kAg_k$
llegamos finalmente a
$$
g_{k+1} = \gamma A^{-1}g_{k+1}
$$
Es decir $g_{k+1}$ es un eigenvector de la matriz $A$.
\par Ahora, por hipótesis, sabemos que $x_{k+2}$ no es un minimizador, así que $g_{k+2}\neq 0$ y el algoritmo no se detendrá en la $k+2$-ésima iteración. Entonces, calculando $\alpha_{k+2} = \frac{g_{k+1}^T\Psi(A)g_{k+1}}{g_{k+1}^T\Psi(A)Ag_{k+1}} = 1/\lambda$ tenemos que $g_{k+3} = (I-\alpha_{k+2}A)g_{k+2} = (1-\alpha_{k+2}\lambda)g_{k+2}=0$, lo que implica que $x_{k+3}$ debe ser el minimizador, por lo que se concluye la prueba.

\par Si tomamos $k_0 =2$ en el teorema, el método utilizando el tamaño de paso dado encontrará al minimizador exacto en máximo 5 iteraciones cuando se minimiza una función cuadrática bidimensional fuertemente convexa.
