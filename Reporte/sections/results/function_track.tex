\subsection*{Estabilidad de resultados}

Se realizaron 100 ejecucciones para cada par de función y método de optimización partiendo de un punto aleatorio dado por la distribución normal con media 0 y desviación 10 (ecuación \ref{eq:initial_point}).

\begin{equation}
    x_0 =  \mathcal{N}(\mu=0;\sigma=10)
    \label{eq:initial_point}
\end{equation}

En la figura \ref{fig:quadracic_function} se muestra el valor de la función y la norma del gradiente para la mejor iteración para cada método con la función cuadrática con matriz diagonal.

\begin{figure}[H]
    \centering
    \includegraphics[width=17cm]{graphics/function/quadratic.png}
    \caption{Valor de la función y norma del gradiente en cada iteración de la mejor ejecución de cada método de optimización de la función cuadrática con matriz diagonal.}
    \label{fig:quadracic_function}
\end{figure}

Se realizó el mismo procedimiento con la función de Rosembrock. En la figura \ref{fig:rosembrock_function} se muestra el resultado del valor de la función y norma del gradiente para la mejor iteración de cada método.

\begin{figure}[H]
    \centering
    \includegraphics[width=17cm]{graphics/function/rosembrock.png}
    \caption{Valor de la función y norma del gradiente en cada iteración de la mejor ejecución de cada método de optimización de la función de Rosembrock con matriz diagonal.}
    \label{fig:rosembrock_function}
\end{figure}

En las tablas \ref{table:mean} y \ref{table:deviation} se muestran las medias y desviaciones estandar para el valor de la función y norma de gradiente en el punto óptimio y el número de iteraciones realizadas por cada método y función.

\begin{table}[H]
    \small
    \centering
    \begin{tabular}{llrrrrr}
        \hline
        \textbf{Función} & \textbf{Valor} & \textbf{SD} & \textbf{BB} & \textbf{ANGM} & \textbf{ANGR1} & \textbf{ANGR2} \\[0.1cm]\hline
                         & Función        & 0.00        & 0.00        & 0.00          & 0.00           & 0.00           \\
        Lambda           & Gradiente      & 0.00        & 0.00        & 0.00          & 0.00           & 0.00           \\
                         & Iteraciones    & 7764.81     & 2770.34     & 252.43        & 289.49         & 255.83         \\\hline
                         & Función        & 0.00        & 0.00        & 0.00          & 0.00           & 0.00           \\
        Cuadrática       & Gradiente      & 0.00        & 0.00        & 0.00          & 0.00           & 0.00           \\
                         & Iteraciones    & 713.00      & 1531.23     & 5.24          & 7.36           & 9.70           \\\hline
                         & Función        & 4.16        & 42.25       & 0.00          & 0.00           & 0.00           \\
        Rosembrock       & Gradiente      & 0.57        & 45.45       & 0.00          & 0.00           & 0.00           \\
                         & Iteraciones    & 10000.00    & 1603.21     & 68.84         & 70.17          & 67.15          \\\hline
                         & Función        & 0.00        & 3.16        & 5.07          & 4.56           & 8.43           \\
        Wood             & Gradiente      & 0.00        & 0.00        & 0.00          & 0.00           & 0.00           \\
                         & Iteraciones    & 9428.69     & 1806.43     & 360.36        & 292.43         & 199.21         \\\hline
    \end{tabular}
    \caption{Media de las 100 ejecucciones partiendo de puntos aleatorios.}
    \label{table:mean}
\end{table}

\begin{table}[H]
    \changefontsizes{11pt}
    \centering
    \begin{tabular}{llrrrrr}
        \hline                                                                                                          \\
        \textbf{Función} & \textbf{Valor} & \textbf{SD} & \textbf{BB} & \textbf{ANGM} & \textbf{ANGR1} & \textbf{ANGR2} \\[0.1cm]\hline
        \\
                         & Función        & 0.000000    & 0.000000    & 0.000000      & 0.000000       & 0.000000       \\[0.25cm]
        Lambda           & Gradiente      & 0.000000    & 0.000000    & 0.000000      & 0.000000       & 0.000000       \\[0.25cm]
                         & Iteraciones    & 612.385412  & 209.340437  & 32.553066     & 28.338322      & 25.043016      \\[0.25cm]\hline
        \\
                         & Función        & 0.000000    & 0.000000    & 0.000000      & 0.000000       & 0.000000       \\[0.25cm]
        Cuadrática       & Gradiente      & 0.000000    & 0.000000    & 0.000000      & 0.000000       & 0.000000       \\[0.25cm]
                         & Iteraciones    & 43.810658   & 96.111137   & 1.198652      & 1.856030       & 2.886751       \\[0.25cm]\hline
        \\
                         & Función        & 14.244453   & 393.585673  & 0.000000      & 0.000000       & 0.000000       \\[0.25cm]
        Rosembrock       & Gradiente      & 1.357743    & 396.058464  & 0.000000      & 0.000000       & 0.000000       \\[0.25cm]
                         & Iteraciones    & 0.000000    & 1867.607211 & 17.317429     & 20.596192      & 14.927011      \\[0.25cm]\hline
        \\
                         & Función        & 0.000000    & 10.092718   & 12.223427     & 11.860350      & 14.404998      \\[0.25cm]
        Wood             & Gradiente      & 0.000019    & 0.000000    & 0.000000      & 0.000000       & 0.000000       \\[0.25cm]
                         & Iteraciones    & 465.597882  & 633.528762  & 190.558643    & 204.762156     & 119.763512     \\[0.25cm]\hline
    \end{tabular}
    \caption{Desviaciones estandar de las 100 ejecucciones partiendo de puntos aleatorios.}
    \label{table:deviation}
\end{table}