\section{Resultados}

\begin{frame}{\insertsectionhead}
    \framesubtitle{Función de Rosembrock}
    La función de Rosembrock se define en la ecuación \ref{eq:rosembrock}.

    \begin{equation}
        f(x) = \sum_{i=1}^{n-1}  100(x_{i+1}-x_{i}^2)^2 +(1-x_i)
        \label{eq:rosembrock}
    \end{equation}
\end{frame}

\begin{frame}{\insertsectionhead}
    \framesubtitle{Función Wood}
    La función de Wood se define en la ecuación \ref{eq:wood}.

    \tiny
    \begin{equation}
        f(x) = 100(x_1^2-x_2)^2+(x_1-1)^2+(x_3-1)^2+90(x_3-x_4)^2 +10.1((x_2-1)^2+(x_4-1)^2)+19.8(x_2-1)(x_4-1) \label{eq:wood}
    \end{equation}
\end{frame}

\begin{frame}{\insertsectionhead}
    \framesubtitle{Función Lambda}
    En el artículo de Yakui Huang se utiliza una función definida de la siguiente manera:

    \begin{equation*}
        f(x) = \frac{1}{2}x^TAx \qquad A=\begin{cases}
            0                           & \text{para } i\neq 0 \\
            10^{\frac{ncond}{n-1}(n-j)} & \text{para } i=j
        \end{cases}
    \end{equation*}

    donde $ncond=log \kappa$ con $\kappa = 10^3$ y $n=10$.
\end{frame}

\begin{frame}{\insertsectionhead}
    \framesubtitle{Función cuadrática}
    En el mismo artículo\cite{huang_2022_1} se propone una función cuadrática donde la matriz esta definida de la siguiente manera:

    \begin{equation*}
        A = diag\{1,\lambda\}
    \end{equation*}

    En nuestro caso tomamos $\lambda=10$.
\end{frame}

\begin{frame}{\insertsectionhead}
    \framesubtitle{Porcentaje de contribución}
    Se definió una función $\gamma$ para medir el porcentaje de las dos componentes más grandes del gradiente en cada iteración de la optimización. La función $\gamma$ tiene la siguiente forma

    \begin{equation*}
        \gamma(g_k) = \frac{|g_k^{(1)}|+|g_k^{(2)}|}{\sum\limits_i |g_k^{(i)}|}
    \end{equation*}

    Realizando el calculo de la función $\gamma$ con la función lambda en el punto inicial $x=(10,10,\dots,10)^T$ se obtuvieron los resultados mostrados en la figura \ref{fig:gamma} para los diferentes métodos.
\end{frame}

\begin{frame}
    \vspace{0.5cm}
    \begin{figure}[H]
        \begin{subfigure}{7cm}
            \includegraphics[width=1\textwidth]{Graphics/gamma/barzilai.png}
            \caption{Método BB.}
        \end{subfigure}
        \begin{subfigure}{7cm}
            \includegraphics[width=1\textwidth]{Graphics/gamma/ANGM.png}
            \caption{Método ANGM.}
        \end{subfigure}
        \begin{subfigure}{7cm}
            \includegraphics[width=1\textwidth]{Graphics/gamma/ANGR1.png}
            \caption{Método ANGR1}
        \end{subfigure}
        \begin{subfigure}{7cm}
            \includegraphics[width=1\textwidth]{Graphics/gamma/ANGR1.png}
            \caption{Método ANGR2.}
        \end{subfigure}
        \caption{\small Función $\gamma$ para diferentes métodos (izquierda) y la posición en el vector de la componente más grande (derecha).}
        \label{fig:gamma}
    \end{figure}
\end{frame}

\begin{frame}{\insertsectionhead}
    \framesubtitle{Porcentaje de contribución}
    \subsection*{Porcentaje de contribución}

Se definió una función $\gamma$ para medir el porcentaje de las dos componentes más grandes del gradiente en cada iteración de la optimización. La función $\gamma$ tiene la siguiente forma

\begin{equation*}
    \gamma(g_k) = \frac{|g_k^{(1)}|+|g_k^{(2)}|}{\sum\limits_i |g_k^{(i)}|}
\end{equation*}

Realizando el calculo de la función $\gamma$ con la función lambda en el punto inicial $x=(10,10,\dots,10)^T$ se obtuvieron los resultados mostrados en la figura \ref{fig:gamma} para los diferentes métodos.

\begin{figure}[H]
    \begin{subfigure}{8.4cm}
        \includegraphics[width=1\textwidth]{graphics/gamma/steepest.png}
        \caption{Método de descenso de gradiente.}
    \end{subfigure}
    \begin{subfigure}{8.4cm}
        \includegraphics[width=1\textwidth]{graphics/gamma/barzilai.png}
        \caption{Método BB.}
    \end{subfigure}
    \begin{subfigure}{8.4cm}
        \includegraphics[width=1\textwidth]{graphics/gamma/ANGM.png}
        \caption{Método ANGM.}
    \end{subfigure}
    \begin{subfigure}{8.4cm}
        \includegraphics[width=1\textwidth]{graphics/gamma/ANGR1.png}
        \caption{Método ANGR1}
    \end{subfigure}
    \begin{subfigure}{16.5cm}
        \centering
        \includegraphics[width=8.4cm]{graphics/gamma/ANGR1.png}
        \caption{Método ANGR2.}
    \end{subfigure}
    \caption{Función $\gamma$ para diferentes métodos (izquierda) y la posición en el vector de la componente más grande (derecha).}
    \label{fig:gamma}
\end{figure}

En la tabla \ref{table:gamma_function} se muestra el número de iteraciones en las que se obtuvo un valor mayor a 0.8 para la función $\gamma$ en cada métodode optimización.

\subsection*{Porcentaje de contribución}

Se definió una función $\gamma$ para medir el porcentaje de las dos componentes más grandes del gradiente en cada iteración de la optimización. La función $\gamma$ tiene la siguiente forma

\begin{equation*}
    \gamma(g_k) = \frac{|g_k^{(1)}|+|g_k^{(2)}|}{\sum\limits_i |g_k^{(i)}|}
\end{equation*}

Realizando el calculo de la función $\gamma$ con la función lambda en el punto inicial $x=(10,10,\dots,10)^T$ se obtuvieron los resultados mostrados en la figura \ref{fig:gamma} para los diferentes métodos.

\begin{figure}[H]
    \begin{subfigure}{8.4cm}
        \includegraphics[width=1\textwidth]{graphics/gamma/steepest.png}
        \caption{Método de descenso de gradiente.}
    \end{subfigure}
    \begin{subfigure}{8.4cm}
        \includegraphics[width=1\textwidth]{graphics/gamma/barzilai.png}
        \caption{Método BB.}
    \end{subfigure}
    \begin{subfigure}{8.4cm}
        \includegraphics[width=1\textwidth]{graphics/gamma/ANGM.png}
        \caption{Método ANGM.}
    \end{subfigure}
    \begin{subfigure}{8.4cm}
        \includegraphics[width=1\textwidth]{graphics/gamma/ANGR1.png}
        \caption{Método ANGR1}
    \end{subfigure}
    \begin{subfigure}{16.5cm}
        \centering
        \includegraphics[width=8.4cm]{graphics/gamma/ANGR1.png}
        \caption{Método ANGR2.}
    \end{subfigure}
    \caption{Función $\gamma$ para diferentes métodos (izquierda) y la posición en el vector de la componente más grande (derecha).}
    \label{fig:gamma}
\end{figure}

En la tabla \ref{table:gamma_function} se muestra el número de iteraciones en las que se obtuvo un valor mayor a 0.8 para la función $\gamma$ en cada métodode optimización.

\subsection*{Porcentaje de contribución}

Se definió una función $\gamma$ para medir el porcentaje de las dos componentes más grandes del gradiente en cada iteración de la optimización. La función $\gamma$ tiene la siguiente forma

\begin{equation*}
    \gamma(g_k) = \frac{|g_k^{(1)}|+|g_k^{(2)}|}{\sum\limits_i |g_k^{(i)}|}
\end{equation*}

Realizando el calculo de la función $\gamma$ con la función lambda en el punto inicial $x=(10,10,\dots,10)^T$ se obtuvieron los resultados mostrados en la figura \ref{fig:gamma} para los diferentes métodos.

\begin{figure}[H]
    \begin{subfigure}{8.4cm}
        \includegraphics[width=1\textwidth]{graphics/gamma/steepest.png}
        \caption{Método de descenso de gradiente.}
    \end{subfigure}
    \begin{subfigure}{8.4cm}
        \includegraphics[width=1\textwidth]{graphics/gamma/barzilai.png}
        \caption{Método BB.}
    \end{subfigure}
    \begin{subfigure}{8.4cm}
        \includegraphics[width=1\textwidth]{graphics/gamma/ANGM.png}
        \caption{Método ANGM.}
    \end{subfigure}
    \begin{subfigure}{8.4cm}
        \includegraphics[width=1\textwidth]{graphics/gamma/ANGR1.png}
        \caption{Método ANGR1}
    \end{subfigure}
    \begin{subfigure}{16.5cm}
        \centering
        \includegraphics[width=8.4cm]{graphics/gamma/ANGR1.png}
        \caption{Método ANGR2.}
    \end{subfigure}
    \caption{Función $\gamma$ para diferentes métodos (izquierda) y la posición en el vector de la componente más grande (derecha).}
    \label{fig:gamma}
\end{figure}

En la tabla \ref{table:gamma_function} se muestra el número de iteraciones en las que se obtuvo un valor mayor a 0.8 para la función $\gamma$ en cada métodode optimización.

\input{sections/results/tables/gamma.tex}
\end{frame}

\begin{frame}{\insertsectionhead}
    \framesubtitle{Estabilidad de resultados}
    \begin{figure}[H]
        \centering
        \includegraphics[width=10cm]{Graphics/function/quadratic.png}
        \caption{Valor de la función y norma del gradiente en cada iteración de la mejor ejecución de cada método de optimización de la función cuadrática con matriz diagonal.}
        \label{fig:quadracic_function}
    \end{figure}
\end{frame}

\begin{frame}{\insertsectionhead}
    \framesubtitle{Estabilidad de resultados}
    \begin{figure}[H]
        \centering
        \includegraphics[width=10cm]{Graphics/function/rosembrock.png}
        \caption{Valor de la función y norma del gradiente en cada iteración de la mejor ejecución de cada método de optimización de la función de Rosembrock con matriz diagonal.}
        \label{fig:rosembrock_function}
    \end{figure}
\end{frame}

\begin{frame}
    \vspace{0.5cm}
    \begin{table}[H]
    \small
    \centering
    \begin{tabular}{llrrrrr}
        \hline
        \textbf{Función} & \textbf{Valor} & \textbf{SD} & \textbf{BB} & \textbf{ANGM} & \textbf{ANGR1} & \textbf{ANGR2} \\[0.1cm]\hline
                         & Función        & 0.00        & 0.00        & 0.00          & 0.00           & 0.00           \\
        Lambda           & Gradiente      & 0.00        & 0.00        & 0.00          & 0.00           & 0.00           \\
                         & Iteraciones    & 7764.81     & 2770.34     & 252.43        & 289.49         & 255.83         \\\hline
                         & Función        & 0.00        & 0.00        & 0.00          & 0.00           & 0.00           \\
        Cuadrática       & Gradiente      & 0.00        & 0.00        & 0.00          & 0.00           & 0.00           \\
                         & Iteraciones    & 713.00      & 1531.23     & 5.24          & 7.36           & 9.70           \\\hline
                         & Función        & 4.16        & 42.25       & 0.00          & 0.00           & 0.00           \\
        Rosembrock       & Gradiente      & 0.57        & 45.45       & 0.00          & 0.00           & 0.00           \\
                         & Iteraciones    & 10000.00    & 1603.21     & 68.84         & 70.17          & 67.15          \\\hline
                         & Función        & 0.00        & 3.16        & 5.07          & 4.56           & 8.43           \\
        Wood             & Gradiente      & 0.00        & 0.00        & 0.00          & 0.00           & 0.00           \\
                         & Iteraciones    & 9428.69     & 1806.43     & 360.36        & 292.43         & 199.21         \\\hline
    \end{tabular}
    \caption{Media de las 100 ejecucciones partiendo de puntos aleatorios.}
    \label{table:mean}
\end{table}
\end{frame}

\begin{frame}
    \vspace{0.5cm}
    \begin{table}[H]
    \small
    \centering
    \begin{tabular}{llrrrrr}
        \hline
        \textbf{Función} & \textbf{Valor} & \textbf{SD} & \textbf{BB} & \textbf{ANGM} & \textbf{ANGR1} & \textbf{ANGR2} \\[0.1cm]\hline
                         & Función        & 0.00        & 0.00        & 0.00          & 0.00           & 0.00           \\
        Lambda           & Gradiente      & 0.00        & 0.00        & 0.00          & 0.00           & 0.00           \\
                         & Iteraciones    & 612.38      & 209.34      & 32.55         & 28.34          & 25.04          \\\hline
                         & Función        & 0.00        & 0.00        & 0.00          & 0.00           & 0.00           \\
        Cuadrática       & Gradiente      & 0.00        & 0.00        & 0.00          & 0.00           & 0.00           \\
                         & Iteraciones    & 43.81       & 96.11       & 1.19          & 1.85           & 2.88           \\\hline
                         & Función        & 14.24       & 393.58      & 0.00          & 0.00           & 0.00           \\
        Rosembrock       & Gradiente      & 1.36        & 396.05      & 0.00          & 0.00           & 0.00           \\
                         & Iteraciones    & 0.00        & 1867.61     & 17.32         & 20.59          & 14.92          \\\hline
                         & Función        & 0.00        & 10.09       & 12.22         & 11.86          & 14.40          \\
        Wood             & Gradiente      & 0.00        & 0.00        & 0.00          & 0.00           & 0.00           \\
                         & Iteraciones    & 465.60      & 633.53      & 190.56        & 204.76         & 119.76         \\\hline
    \end{tabular}
    \caption{Desviaciones estandar de las 100 ejecucciones partiendo de puntos aleatorios.}
    \label{table:deviation}
\end{table}
\end{frame}